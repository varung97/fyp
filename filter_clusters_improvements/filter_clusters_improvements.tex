\documentclass{article}

\usepackage{amsmath}
\usepackage{amssymb}
\usepackage{enumerate}
\usepackage{relsize}
\usepackage{algorithm}
\usepackage{algpseudocode}

\usepackage[
  top    = 2.75cm,
  bottom = 3.00cm,
  left   = 2.50cm,
  right  = 2.50cm
]{geometry}
\setlength{\parindent}{0in}

\title{\texttt{Filter\_Clusters} improvements}

\begin{document}
    \maketitle

    \section{Preliminaries}
    Define:
    \begin{itemize}
        \item $T[u]$ for some tree $T$ and some node $u$ to be the subtree of $T$ rooted at $u$
        \item $\Lambda(T)$ for some tree $T$ to be the leaf set of $T$
        \item $T|C$ to be the tree $T$ restricted to the leaf set $C$, i.e. keep all common ancestors of $C$ while keeping the relative structure of these nodes the same as in the original tree
        \item $T||C$ to be a modified version of $T|C$. This version includes ``special nodes''. Given a parent-child pair in $T|C$ a special node is added between them if these nodes are not a parent-child pair in $T$; this special node represents all the nodes on the path between the pair in $T$
        \item Trees $T_A$ and $T_B$ on which the procedure \texttt{Filter\_Clusters} will be run
        \item Centroid path of $T_A$ to be $\pi$. Number the nodes from leaf to root from $1$ to $\alpha$
        \item Set of side trees created when $\pi$ is removed from $T_A$ to be $\sigma(\pi)$
        \item $k$ to be the total number of trees over which the frequency difference tree is being computed
        \item $n$ to be the size of the leaf set of each of these trees (each of which is the ,,,same)
        \item weight of a node $u$ in one of the trees under consideration to be the number of trees that its associated cluster appears in
    \end{itemize}

    \section{Bottlenecks}

    Note that there are two bottlenecks in \texttt{Filter\_Clusters}:
    \begin{itemize}
        \item Step 3 preprocesses $T_B$ in $O(n\;log\;n)$ time and then uses $\sum_{\tau\in\sigma(\pi)}O(|\Lambda(\tau)|log(|\Lambda(\tau)|))$ time to construct each $T_B||\Lambda(\tau)$.

        \item Step 6 involves making $O(n)$ add/remove operations on a BT of size $O(n)$, costing $O(n\;log\;n)$
    \end{itemize}

    \section{Step 3}

    First, to avoid preprocessing $T_B$ for RMQ queries within each recursive call, simply do it once at the beginning. We will now discuss how this original tree can be used to answer RMQ queries in recursive calls.\\

    Note that the RMQ structure can give us a handle to the node which has the max weight between two nodes. For some side tree $\tau$, when constructing $T_B||\Lambda(\tau)$ we add some special nodes $z$ to $T_B|\Lambda(\tau)$ which represent a node with max weight between some two nodes. These nodes should keep handles to the node in the original tree that they identify with. In recursive calls, if asked to construct a special node where one or both of the endpoints is/are special node(s), use the handle to the node in the original tree to make this query. That is, all RMQ queries are made against the original preprocessed tree.\\

    Constructing each $T_B|\Lambda(\tau)$ costs total $O(n)$ time [Original paper]. From each of these, contructing $T_B||\Lambda(\tau)$ will then cost $O(|\Lambda(\tau)|)$ time each since each tree contains $O(|\Lambda(\tau)|)$ edges and each RMQ query costs $O(1)$. Also, marking spoiled nodes can be done in $O(|\Lambda(\tau)|)$ time per $\tau$, by doing a bottom up traversal of the tree and counting the size of the leaf set.\\

    Thus step 3 can be completed in $O(n)$ time where $n$ is the size of the subproblem, along with a single preprocessing step at the start of the \texttt{Filter\_Clusters} subroutine, costing $O(n\;log\;n)$ time, where $n$ is the size of the leaf set.

    \section{Step 6}

    \begin{algorithm}
        \caption{Filter\_Clusters}
        \begin{algorithmic}[1]
            \State \textbf{Input:} Two trees $T_A$, $T_B$ with $\Lambda(T_A) = \Lambda(T_B) = L$ where every cluster has a known $weight$.

            \State \textbf{Output:} A tree $T$ with $\Lambda(T) = L$ and $C(T) = \{\Lambda(T_A[u]): u \in V(T_A) \text{ and } w(u) > w(x) \text{ for every } x \in V(T_B) \text{ with } \Lambda(T_A[u]) \text{ incompatible with } \Lambda(T_B[x])\}$

            \State Compute a centroid path $\pi = \;<p_{\alpha}, p_{\alpha - 1},...,p_1>$ of $T_A$, where $p_{\alpha}$ is the root of $T_A$ and $p_1$ is a leaf.

            \State For each side tree $\tau \in \sigma(\pi)$ construct $T_B||\tau$.

            \State Let $R_s$ be a tree consisting only of a root node and a single leaf labelled by $p_1$.

            \State Let $R_c$ be a tree with $\Lambda(R_c) = L$ where every leaf is attached to the root.

            \State Do a bottom up traversal of $T_B$ to precompute $|\Lambda(T_B[x])|$ for every $x \in V(T_B)$. Preprocess $T_B$ for $lca$ queries. Let $IDS$ be an initially empty Brodal queue for storing nodes from $T_B$. For every $x \in V(T_B)$, initialise $counter(x) := 0$.

            \State Let $r_i :=$ the leaf in $T_B$ labelled by $p_1$.\newline
            Set $counter(r_1) := 1$, and if $\alpha \geq 2$ then $counter(parent^{T_B}(r_1)) := 1$.

            \Comment{Continued on next page}

            \algstore{FC}
        \end{algorithmic}
    \end{algorithm}

    \begin{algorithm}
        \begin{algorithmic}[1]
            \algrestore{FC}
            \For{$i := 2$ \textbf{to} $\alpha$}
                \State Let $V$, $parentsOfSubtrees :=$ empty linked list.

                \ForAll{side trees $\tau$ associated with $p_i$}
                    \State $\tau'$, $V' := $ \texttt{Filter\_Clusters}($\tau$, $T_B||\tau$)

                    \State Attach $root(\tau')$ as a child of $root(R_s)$

                    \If{$V'$ is empty}
                        \State Add $parent^{T_B}(root(T_B||\tau))$ to $V$
                    \Else
                        \State $V :=$ \texttt{Concatenate}($V$, $V'$)
                    \EndIf
                \EndFor

                \State Compute $r_i := lca^{T_B}(r_{i-1} \cup V)$

                \State Insert every node belonging to the path between $r_{i-1}$ and $r_i$, except $r_{i-1}$ and $r_i$, into $IDS$

                \If{$counter(r_{i-1}) < |\Lambda(T_B[r_{i-1}])|$ or $r_{i-1}$ is spoiled}
                    \State Insert $r_{i-1}$ into $IDS$
                \EndIf

                \ForAll{$x \in V$}
                    \While{$x$ is not in $IDS$ and $x \neq r_i$}
                        \State Insert $x$ into $D$
                        \State $x := parent^{T_B}(x)$
                    \EndWhile
                \EndFor

                \ForAll{$x \in V$}
                    \State $counter(x) := counter(x) + 1$

                    \While{$counter(x) = |\Lambda(T_B[x])|$ and $x$ is not spoiled and $x \neq r_i$}
                        \State $counter(parent^{T_B}(x)) := counter(parent^{T_B}(x)) + |\Lambda(T_B[x])|$
                        \State Remove $x$ from $IDS$
                        \If{$x$ is in $parentsOfSubtrees$}
                            \State Remove $x$ from $parentsOfSubtrees$
                        \EndIf
                        \State $x := parent^{T_B}(x)$
                    \EndWhile

                    \If{$x$ is not in $parentsOfSubtrees$ and $x \neq root(T_B)$}
                        \State Add $x$ to $parentsOfSubtrees$
                    \EndIf
                \EndFor

                \If{$IDS$ is empty}
                    \State $M := 0$
                \Else
                    \State $M :=$ maximum weight of a node in $IDS$
                \EndIf

                \If{$w(\Lambda(T_A[p_i])) > M$}
                    \State Put $\Lambda(T_A[p_i])$ in $R_c$ by an $insert$ operation
                \EndIf
            \EndFor

            \State $T := $ \texttt{Merge\_Trees}($R_s$, $R_c$)

            \State \Return $T$, $parentsOfSubtrees$
        \end{algorithmic}
    \end{algorithm}

    Let the data structure being used to store the incompatible nodes be $IDS$.\\

    \textbf{Lemma 1:} Let $\tau \in \sigma(\pi)$ be some side tree in $T_A$. For any node $u$ that was removed from $IDS$ during the call to \texttt{Filter\_Clusters}$(\tau, T_B||\tau)$, $\Lambda(T_B[u]) \subseteq \Lambda(\tau)$.\\
    \textit{Proof:} Let the centroid path of $\tau$ be $\pi(\tau)$. Notice that when a node $u$ is removed from $IDS$, that means that for some node $q \in \pi(\tau)$, $counter(u) = |\Lambda(T_B||\tau[u]) \cap \Lambda(\tau[q])| = |\Lambda(T_B||\tau[u])|$ and $u$ is not spoiled. Since $u$ is not spoiled, $\Lambda(T_B||\tau[u]) = \Lambda(T_B[u])$. Thus $T_B[u] \subseteq \Lambda(\tau[q])$. Since $\Lambda(\tau[q]) \subseteq \Lambda(\tau)$, $T_B[u] \subseteq \Lambda(\tau)$.\\

    \textbf{Lemma 2:} If for some cluster $C$ and some side tree $\tau \in \sigma(\pi)$, $C \subseteq \Lambda(\tau)$ then $C$ is compatible with $T_A$.\\
    \textit{Proof:} Since the leaf sets of all sidetrees are mutually disjoint, and $C \subseteq \tau$, $C$ is disjoint with leaf sets of all other sidetrees, and so is compatible with all nodes in these. The only remaining nodes are those on the centroid path. Let $p_i \in \pi$ be the node on the centroid path that is a direct parent of $\tau$. Then for any $j$ such that $1 \leq j < i$, $\Lambda(T_A[p_j])$ is mutually disjoint with $C$ and so these are compatible. For any $j$ such that $i \leq j \leq \alpha$, $\Lambda(T_A[p_i]) \subseteq \Lambda(T_A[p_j])$. Also, $\Lambda(\tau) \subseteq \Lambda(T_A[p_i])$. Thus $C$ is compatible with $p_j$.\\

    From Lemma 1 and 2, for any node $u$ and side tree $\tau \in \sigma(\pi)$, if $u$ was removed from $IDS$ during the call to \texttt{Filter\_Clusters}$(\tau, T_B||\tau)$, $\Lambda(T_B[u])$ is compatible with $T_A$. Then, we should never need to add $u$ to $IDS$ again. We detail below how this is ensured.\\

    Firstly, note that the nodes $u$ that are removed from $IDS$ during the recursive calls to \texttt{Filter\_Clusters} form disjoint subtrees of $T_B$ since their leaf sets are subsets of disjoint side trees. For each of these subtrees, we do not wish to iterate over the nodes in them again. Thus, we would like start from the parents of the roots of these trees (rather than the leaves as in the original algorithm). To do so, each call to \texttt{Filter\_Clusters} returns the parents of the roots of the subtrees discovered during its execution as a linked list.\\

    To achieve this, for some side tree $\tau \in \sigma(\pi)$, during the recursive call to \texttt{Filter\_Clusters}$(\tau, T_B||\tau)$, we augment each node in $T_B||\tau$ with a property $parentOfSubtree$, initialised to $False$. This property is used to check membership in an initially empty linked list $parentsOfSubtrees$, and is updated every time a node is inserted/deleted from this list. When traversing $T_B||\tau$ upwards while removing nodes from $IDS$, for every removed node $x$, if $x$ is in $parentsOfSubtrees$ then $x$ is removed from $parentsOfSubtrees$. When the loop terminates, if $x$ is not $root(T_B||\tau)$ and is not in $parentsOfSubtrees$, then $x$ is added to $parentsOfSubtrees$. Notice that at the end of this process $parentsOfSubtrees$ contains exactly the parent of every desired subtree. The linked list may also be empty, indicating that $\Lambda(lca^{T_B}(\Lambda(\tau))) = \Lambda(\tau)$.\\

    This linked list is returned to the caller of \texttt{Filter\_Clusters}. Here, for any $p_i \in \pi$, the linked lists returned by the call to \texttt{Filter\_Clusters} on side trees associated with it are merged by linking them together, and this merged list is used instead of the set $D = \Lambda(T_A[p_i]) \;\backslash\; \Lambda(T_A[p_{i-1}])$. Note that if the linked list for some $\tau \in \sigma(\pi)$ is empty, then we can replace it with a single node $= parent^{T_B}(root(T_B||\tau))$.\\

    Thus a node that is removed from some $IDS$ is never added to an $IDS$ again. Notice that in a call to \texttt{Filter\_Clusters} for a leaf set of size $m$, we might still insert $O(m)$ nodes into the $IDS$. We maintain another property in each node $u$ called $inIDS(u)$, which is set to $True$ when a node is added to the $IDS$ and to $False$ when it is removed. This allows us to make membership checks in $O(1)$ time.\\

    Since we are making fewer deletions than insertions, we use a Brodal queue for $IDS$. This supports insertion and findMax in $O(1)$ time and delete in $O(log\;n)$ time where $n$ is the number of elements in the queue. Then for each call to \texttt{Filter\_Clusters}, total insertion cost is $O(m)$, while total deletion cost over all calls to \texttt{Filter\_Clusters} is $O(n\;log\;n)$

    \section{Overall analysis}

    For \texttt{Filter\_Clusters}, if we don't take the deletion times into account, then we get the recurrence,

    \[T(m) = O(m) + \sum_{\tau\in\sigma(\pi)}T(|\Lambda(\tau)|)\]

    Since there are $O(log\;n)$ recursion levels,
    \[T(n) = O(n\;log\;n)\]

    The total deletion costs is $O(n\;log\;n)$, so in total,

    \[T(n) = O(n\;log\;n)\]
\end{document}
