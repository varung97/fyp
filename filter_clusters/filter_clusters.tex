\documentclass[a4paper]{article}

\usepackage{amsmath}
\usepackage{amssymb}
\usepackage{enumerate}
\usepackage{relsize}
\usepackage[
  top    = 2.75cm,
  bottom = 3.00cm,
  left   = 2.50cm,
  right  = 2.50cm
]{geometry}
\setlength{\parindent}{0in}

\title{\texttt{Filter\_Clusters} runtime}

\begin{document}
    \maketitle

    Note that:
    \begin{itemize}
        \item $\Lambda(T)$ = leaf set of $T$
        \item $|\Lambda(A)| = |\Lambda(B)| = n$
        \item $\sigma(\pi) = $ set of side trees of A (after centroid decomposition)
        \item $T|C$ is the tree constructed by restricting $T$ to the leaf set $C$
        \item $T||C$ is the tree constructed from $T|C$ by adding ``special nodes''
    \end{itemize}

    \vspace*{\bigskipamount}

    From the FDCT paper, runtime of \texttt{Filter\_Clusters}(A, B) can be written as:

    \begin{equation}
        T(\texttt{Filter\_Clusters}(A, B)) = O(n\;logn)\;+\;\sum_{\tau\in\sigma(\pi)} T(\texttt{Filter\_Clusters}(\tau, B||\Lambda(\tau)))
    \end{equation}

    \textbf{Original analysis:}\\

    Each $\tau$ is such that $|\Lambda(\tau)| \leq n/2$, thus there are $log(n)$ recursion levels, and so total runtime $= O(n\;log^2n)$\\

    \textbf{Altered analysis:}\\

    \[\sum\limits_{\tau\in\sigma(\pi)}|\Lambda(\tau)| = n - 1\]\\
    Also, $|B||\Lambda(\tau)| = O(|\Lambda(\tau)|)$ [since $B||\Lambda(\tau)$ is constructed by adding at most one special node per node in $B|\Lambda(\tau)$]\\

    Then, size of subproblem $\texttt{Filter\_Clusters}(\tau, B||\Lambda(\tau)) = O(\Lambda(\tau))$\\

    Thus
    \begin{equation}
        T(\texttt{Filter\_Clusters}(A, B)) = T(n) = O(n\;logn)\;+\;\sum_{\tau\in\sigma(\pi)}T(|\Lambda(\tau)|)
    \end{equation}

    Given the result that if $\sum{m} = n$ then $\sum{m\;logm} = O(n\;logn)$ and inductively assuming that $T(i) = O(i\;logi)$
    \begin{equation}
        T(n) = O(n\;logn)
    \end{equation}
\end{document}
